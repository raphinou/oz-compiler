%http://vim-latex.sourceforge.net/documentation/latex-suite.html#environment-mappings
\documentclass[draft,a4paper]{memoir}
\usepackage[utf8]{inputenc}
\usepackage{pgf}
\usepackage[margin=false,inline=true]{fixme}
\fxsetup{theme=color}
\definecolor{fxnote}{rgb}{0.8000,0.0000,0.0000}
\author{Bauduin Raphaël}
\title{Oz Compiler}
\begin{document}

% requires memoir
% include git rev
\ifdraftdoc
\makeoddhead{plain}{}{}{\textit{Draft: \today{} Rev: \GITAbrHash{} \VCModifiedText{}}}{}
\fi

% write and require version control info
\immediate\write18{sh ./vc -m} 
%%% This file has been generated by the vc bundle for TeX.
%%% Do not edit this file!
%%%
%%% Define Git specific macros.
\gdef\GITHash{166c14995c66d3f18d28e1e99ea5e6f485604b6e}%
\gdef\GITAbrHash{166c149}%
\gdef\GITParentHashes{95e1d413565078e5f269c7f65b42db0ebcfaa4de}%
\gdef\GITAbrParentHashes{95e1d41}%
\gdef\GITAuthorName{Raphaël Bauduin}%
\gdef\GITAuthorEmail{rb@raphinou.com}%
\gdef\GITAuthorDate{2013-02-05 09:13:08 +0100}%
\gdef\GITCommitterName{Raphaël Bauduin}%
\gdef\GITCommitterEmail{rb@raphinou.com}%
\gdef\GITCommitterDate{2013-02-05 09:13:08 +0100}%
%%% Define generic version control macros.
\gdef\VCRevision{\GITAbrHash}%
\gdef\VCAuthor{\GITAuthorName}%
\gdef\VCDateRAW{2013-02-05}%
\gdef\VCDateISO{2013-02-05}%
\gdef\VCDateTEX{2013/02/05}%
\gdef\VCTime{09:13:08 +0100}%
\gdef\VCModifiedText{\textcolor{red}{with local modifications!}}%
%%% Assume clean working copy.
\gdef\VCModified{0}%
\gdef\VCRevisionMod{\VCRevision}%

\maketitle
\tableofcontents


\chapter{Introduction}
\section{Initial State}
New vm developed.
Old compiler patched, boot compiler in scala.
Need for a new Oz compiler, written in OZ

\section{Goal and Scope}
Flexible documented and tested compiler taking as input the AST in the form of Oz records, and generating opcodes send to the assembler.
Document as much as possible the whole compilation chain from parser through AST to opcodes and virtual machine.

\chapter{Infrastructure}
\section{Virtual Machine}
Describe virtual machine. I think to put it first so all concepts are introduced by the time I describe the opcodes.
\subsection{Registers}
Y are locals(y(index)), X are work registers (x(index)), G are globals filled sequentially with arrayFill (g(index)), k are constants (k(value))
\subsection{Abstractions}
Abstraction (with G regs)  -> CodeArea (with K regs) -> Code
\section{Compiler input}
Describe records received from parser.
fInt,fFloat,fVar, fEq, fProc\ldots
\section{Compiler output}
Describe opcodes.


\chapter{Compiler}
\section{Architecture}
Explaine the phases
\subsection{Namer}
replaces fVar by fSym. In declarations, creates a new symbol for declared variables, in body, looks at already defined symbols.
\subsection{Globaliser}
Each scope is assigned a name, and variables are assigned the scope in which they are declared. When a variable is used in a scope other than its declaration scope, it is a global variable relative to its use scope. We replace this symbol by new symbol marked as 'global' referencing the original symbol.

fProc is replaced by fDefineProc taking 2 additional arguments: the global variables and their corresponding new symbols. \fxnote{Check that the two arguments are really needed}
\subsection{CodeGen}
\subsubsection{Register allocation}
It is CodeGen that allocates registers. For the moment only Y registers.
When a Symbol of type global is found, it is handled differntly than local symbols.
For a procedure definition, we initialise the global variables with arrayFill( y(yindex)). \fxnote{handle multiple levels of nesting}
For a local\ldots in \ldots end, a g() is issued in place of a y(). \fxnote{is this working yet?}

\section{Tests}
Explain the tests

\section{Performance}
Possibly include some measure?



\end{document}
